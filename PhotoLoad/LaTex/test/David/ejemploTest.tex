% !TeX spellcheck = es_ES
\documentclass{scrartcl}
%\documentclass{book}
\usepackage[spanish]{babel}
\usepackage[utf8]{inputenc}
\usepackage{enumerate}
\usepackage{graphicx}
\usepackage{color}
\usepackage{float}
\usepackage{multicol}
\usepackage{lscape}
\usepackage{fancyhdr}
\usepackage{tabularx}
\usepackage[hidelinks,colorlinks=true, linkcolor=black]{hyperref} %hyperref para hacer los links del indice y usar \url{URL} y \href{URL}{text}y hidelinks para que no se rodeen con una caja de color los enlaces. Más información en: http://en.wikibooks.org/wiki/LaTeX/Hyperlinks
\pagestyle{fancy}
\begin{document}


\subsection{Prueba GD1}

\begin{tabularx}{14cm}{|c|X|}
	\hline \textbf{ID} & Prueba GD1 \\ 
	\hline \textbf{Descripción} & Comprueba el correcto funcionamiento al solicitar la lista de archivos a la cuenta de Google Drive logeada. \\	 
	\hline  \textbf{Entrada}		& Token (OAuth) \\ 
	\hline  \textbf{Salida esperada}			& Lista de archivos del Google Drive del usuario \\
	\hline  \textbf{Resultado}			& Éxito	 \\
	\hline 
\end{tabularx} 

\subsection{Prueba GD2}

\begin{tabularx}{14cm}{|c|X|}
	\hline \textbf{ID} & Prueba GD2 \\ 
	\hline \textbf{Descripción} & Comprueba el correcto funcionamiento al intentar descargar un archivo de la cuenta de Google Drive logeada.\\	 
	\hline  \textbf{Entrada}		& Token(OAuth) e índice del archivo a descargar en la lista \\ 
	\hline  \textbf{Salida esperada}			& URL para descargar dicho archivo \\
	\hline  \textbf{Resultado}			& Éxito \\
	\hline 
\end{tabularx} 
\end{document}