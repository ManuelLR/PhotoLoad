% !TeX spellcheck = es_ES
\documentclass{scrartcl}
%\documentclass{book}
\usepackage[spanish]{babel}
\usepackage[utf8]{inputenc}
\usepackage{enumerate}
\usepackage{graphicx}
\usepackage{color}
\usepackage{float}
\usepackage{multicol}
\usepackage{lscape}
\usepackage{fancyhdr}
\usepackage{tabularx}
\usepackage[hidelinks,colorlinks=true, linkcolor=black]{hyperref} %hyperref para hacer los links del indice y usar \url{URL} y \href{URL}{text}y hidelinks para que no se rodeen con una caja de color los enlaces. Más información en: http://en.wikibooks.org/wiki/LaTeX/Hyperlinks
\pagestyle{fancy}
\begin{document}


\subsection{Prueba FL1}

\begin{tabularx}{14cm}{|c|X|}
	\hline \textbf{ID} & Prueba FL1 \\ 
	\hline \textbf{Descripción} & Comprueba el correcto funcionamiento de las dos fases de login necesarios para Flickr \\	 
	\hline  \textbf{Entrada}		& No entra ningún dato \\ 
	\hline  \textbf{Salida esperada}			& Un objeto de tipo OAuth con una URL y un AccessToken relleno\\
	\hline  \textbf{Resultado}			& Exito \\
	\hline 
\end{tabularx} 

\subsection{Prueba FL2}

\begin{tabularx}{14cm}{|c|X|}
	\hline \textbf{ID} & Prueba FL2 \\ 
	\hline \textbf{Descripción} & Comprueba la correcta obtención de las fotos existentes en Flickr para el usuario \\	 
	\hline  \textbf{Entrada}		& Token (OAuth) \\ 
	\hline  \textbf{Salida esperada}			& Lista de FlickrPhoto no null con los ids de las fotos pero sin los links de descarga\\
	\hline  \textbf{Resultado}			& Exito \\
	\hline 
\end{tabularx} 

\subsection{Prueba FL3}

\begin{tabularx}{14cm}{|c|X|}
	\hline \textbf{ID} & Prueba FL3\\ 
	\hline \textbf{Descripción} & Comprueba la correcta obtención de los links de las fotos existentes en Flickr para el usuario \\	 
	\hline  \textbf{Entrada}		& Token (OAuth) y lista de FlickrPhoto de las que queremos obtener el link \\ 
	\hline  \textbf{Salida esperada}			& Lista de FlickrPhoto con la información anterior además de links de descarga\\
	\hline  \textbf{Resultado}			& Exito \\
	\hline 
\end{tabularx} 

\subsection{Prueba FB1}

\begin{tabularx}{14cm}{|c|X|}
	\hline \textbf{ID} & Prueba FB1\\ 
	\hline \textbf{Descripción} & Comprueba la correcta obtención de las fotos existentes en Facebook para el usuario \\	 
	\hline  \textbf{Entrada}		& Token \\ 
	\hline  \textbf{Salida esperada}			& Lista de Fotos\\
	\hline  \textbf{Resultado}			& Exito \\
	\hline 
\end{tabularx} 

\subsection{Prueba FB2}

\begin{tabularx}{14cm}{|c|X|}
	\hline \textbf{ID} & Prueba FB2\\ 
	\hline \textbf{Descripción} & Comprueba la correcta subida de fotos al tablon de Facebook \\	 
	\hline  \textbf{Entrada}		& Token y url de la foto a subir \\ 
	\hline  \textbf{Salida esperada}			& El id de la foto subida\\
	\hline  \textbf{Resultado}			& Exito \\
	\hline 
\end{tabularx} 
\end{document}